\labeledsection{Machine-Oriented Calculi for Propositional Logic}{sec:atp_pl0}
\commandnote{From \Thmref{thm:unsat}, we observe that \linkterm{entailment}{entailment_pl0} can be tested via \linkterm{satisfiablity}{satisfiable_ls}}

\definition{Theorem Proving}{
Given a \linkterm{formal system}{formal_system} $\tuple{\mathcal{S}, \mathcal{C}}$, where $\mathcal{C}$ is a \linkterm{calculus}{calculus_logic} and $\mathcal{S} := \langle \mathcal{L}, \mathcal{M}, \vDash \rangle$ is a \linkterm{logical system}{logical_system}, the task of \textbf{theorem proving} is the task of determining whether $\mathcal{H} \vdash_{\mathcal{C}} C$ for a \textbf{conjecture} $C \in \mathcal{L}$ and hypotheses $\mathcal{H} \subseteq \mathcal{L}$
}{theorem_proving}

\definition{Automated Theorem Proving}{
\textbf{ATP} is the automation of \linkterm{theorem proving}{theorem_proving}.
}{def:atp}

\definition{Test Calculus}{
For a given \linkterm{conjecture}{theorem_proving} $A$ and \linkterm{hypotheses}{theorem_proving} $\mathcal{H}$, a \textbf{test calculus} $\mathcal{T}$ tries to derive a \linkterm{refutation}{c_refutation} $\mathcal{H}, \neg A \vdash_{\mathcal{T}} \bot$ instead of $\mathcal{H} \vdash A$, where $\neg A$ is \linkterm{unsatisfiable}{unsatisfiable_ls} iff $A$ is \linkterm{valid}{valid_ls} and $\bot$ an \linkterm{unsatisfiable}{unsatisfiable_ls} \linkterm{proposition}{proposition}.
}{test_calculus}


\definition{Atomic Formulae}{
A \linkterm{formula}{formulae} is called \textbf{atmoic} (or an \textbf{atom}) if it does not contain logical constants, otherwise, \textbf{complex}
}{atomic_formula}

\definition{Labeled Formula}{
Let $\mathcal{S} := \langle \mathcal{L}, \mathcal{M}, \vDash \rangle$ be a \linkterm{logical system}{logical_system}, $A \in \mathcal{L}$ a formula, $L$ a \textbf{label set}, and $\alpha \in L$ a \textbf{label}, then we call a \linkterm{pair}{cartesian_product} $\tuple{A, \alpha}$ a \textbf{labeled formula} and write it as $A^{\alpha}$. For a \linkterm{set}{def:set} $\Phi$ of \linkterm{propositions}{proposition} we use $\Phi^{\alpha} := \set{A^{\alpha} \mid A \in \Phi}$
}{labeled_formula}

\commandnote{
If the \linkterm{label set}{labeled_formula} is $\mathcal{D}_0 := \set{T,F}$, we call a \linkterm{labeled formula}{labeled_formula} $A^T$ \textbf{positive} and $A^F$ \textbf{negative}
}

\commandnote{
Let $\mathcal{S} := \langle \mathcal{L}, \mathcal{M}, \vDash \rangle$ be a \linkterm{logical system}{logical_system}, and $A^{\alpha}$ a \linkterm{labeled formula}{labeled_formula}. Then we say that $M \in \mathcal{M}$ \textbf{satisfies} $A^{\alpha}$ (write $M \vDash A^{\alpha}$), iff $\alpha = T$ and $M \vDash A$ or $\alpha = F$ and $M \nvDash A$
}

\definition{Literal}{
Let $\mathcal{S} := \langle \mathcal{L}, \mathcal{M}, \vDash \rangle$ be a \linkterm{logical system}{logical_system}, $A \in \mathcal{L}$ is an \linkterm{atomic formula}{atomic_formula}, and $\alpha \in \set{T,F}$, then we call the \linkterm{labeled formula}{labeled_formula} $A^{\alpha}$ a \textbf{literal}.
}{literal}

\definition{Opposite Literal}{
For a \linkterm{literal}{literal} $A^{\alpha}$, we call the \linkterm{literal}{literal} $A^{\beta}$ with $\alpha \neq \beta$ the \textbf{opposite literal}.
}{opposite_literal}

\definition{CNF}{
A \linkterm{formula}{wff0_def} is in \textbf{conjunctive normal form (CNF)} if it is $\top$ or a conjunction of disjunctions of \linkterm{literals}{literal}, i.e. if it is of the form:
\[
\bigwedge_{i=1}^{n} \bigvee_{j=1}^{m_i} l_{ij} := 
\underbrace{
    \overbrace{( l_{1,1} \lor \dots \lor l_{1,m_1} )}^{\text{Clause } i=1 \text{ has } m_1 \text{ literals}}
    \land
    \overbrace{( l_{2,1} \lor \dots \lor l_{2,m_2} )}^{\text{Clause } i=2 \text{ has } m_2 \text{ literals}}
    \land \dots \land
    \overbrace{( l_{n,1} \lor \dots \lor l_{n,m_n} )}^{\text{Clause } i=n \text{ has } m_n \text{ literals}}
}_{\text{There are } n \text{ total clauses}}
\]
}{cnf}

\definition{DNF}{
A \linkterm{formula}{wff0_def} is in \textbf{disjunctive normal form (DNF)} if it is $\bot$ or a disjunction of conjunctions of \linkterm{literals}{literal}, i.e. if it is of the form:
\[
\bigvee_{i=1}^{n} \bigwedge_{j=1}^{m_i} l_{ij} := 
\underbrace{
    \overbrace{( l_{1,1} \land \dots \land l_{1,m_1} )}^{\text{Term } i=1 \text{ has } m_1 \text{ literals}}
    \lor
    \overbrace{( l_{2,1} \land \dots \land l_{2,m_2} )}^{\text{Term } i=2 \text{ has } m_2 \text{ literals}}
    \lor \dots \lor
    \overbrace{( l_{n,1} \land \dots \land l_{n,m_n} )}^{\text{Term } i=n \text{ has } m_n \text{ literals}}
}_{\text{There are } n \text{ total terms}}
\]
}{dnf}


\commandnote{
$\top$ is in CNF because it represents the \textbf{empty conjunction}. (The semantic identity for conjunction is $T$, so we use the constant $\top$ where $\mathcal{I}(\top)=T$).
$\bot$ is in DNF because it represents the \textbf{empty disjunction}. (The semantic identity for disjunction is $F$, so we use the constant $\bot$ where $\mathcal{I}(\bot)=F$).
}

\definition{Tableau Calculus}{
A \textbf{tableau calculus} is a \linkterm{test calculus}{test_calculus} that analyzes a \linkterm{labeled formula}{labeled_formula} in a \linkterm{tree}{tree} to determine \linkterm{satisfiability}{satisfiable_ls}, its branches correspond to \linkterm{valuations}{value_function_pl0} (\linkterm{models}{model_pl0})
}{tableau_calculus}

\definition{Saturated Tableau}{
We call a tableau \textbf{saturated} iff no application of a \linkterm{rule}{inference_rules_logic} adds new material to any branch.
}{saturated_tableau}

\definition{Closed and Open Branches}{
A tableau branch is \textbf{closed} if it ends with $\bot$. Otherwise, the branch is \textbf{open}.
}{closed_branch}

\commandnote{
We sometimes decorate \textbf{open} branches with a $\Box$ symbol to visually indicate they are finished and non-contradictory.
}

\definition{Closed Tableau}{
A tableau is \textbf{closed} iff every one of its branches is \linkterm{closed}{closed_branch}. Otherwise, the tableau is \textbf{open}.
}{closed_tableau}

\commandnote{
It is crucial to distinguish between the \textbf{outcome} of a proof and the \textbf{completion} of the process:
\begin{itemize}
    \item \textbf{Closed vs. Open:} This refers to the \textit{logical status}. A tableau is \textbf{closed} if a contradiction is found on \textbf{every} branch. It is \textbf{open} if at least one branch remains consistent.
    \item \textbf{Saturated vs. Unsaturated:} This refers to the \textit{construction status}. A tableau is \textbf{saturated} if no more rules can be applied to any branch.
\end{itemize}
\textbf{Note:} A \textbf{closed} tableau is rarely \textbf{saturated}. Once a specific branch closes, we stop applying rules to \textbf{that branch} for efficiency, even if formulas on that branch could still be expanded.
}

\commandnote{
\linkterm{Open branches}{closed_branch} in a \linkterm{saturated tableaux}{saturated_tableau} yield \linkterm{satisfiying}{satisfiable_ls} \linkterm{assignments}{var_ass_pl0}
}

\definition{$\mathcal{T}_0$}{
The Propositional Tableau Calculus $\mathcal{T}_0$ has the \linkterm{inference rules}{inference_rules_logic} shown in \figref{fig:t0_calculus}
}{t0_calculus}

\begin{figure}[H]
    \centering
    \begin{tcolorbox}[colback=white, colframe=black, sharp corners, boxrule=0.5pt]
        \begin{multicols}{2}
            \noindent
            \begin{minipage}{\linewidth}
                \[
                    \infer[\mathcal{T}_0 \land]{
                        \begin{array}{c}
                            A^T \\
                            B^T
                        \end{array}
                    }{(A \land B)^T}
                \]
            \end{minipage}

            \noindent
            \begin{minipage}{\linewidth}
                \[
                    \infer[\mathcal{T}_0 \lor]{A^F \mid B^F}{(A \land B)^F}
                \]
            \end{minipage}
        \end{multicols}

        \begin{multicols}{2}
            % Negation True Rule
            \noindent
            \begin{minipage}{\linewidth}
                \[
                    \infer[\mathcal{T}_0 \neg^T]{A^F}{(\neg A)^T}
                \]
            \end{minipage}

            % Negation False Rule
            \noindent
            \begin{minipage}{\linewidth}
                \[
                    \infer[\mathcal{T}_0 \neg^F]{A^T}{(\neg A)^F}
                \]
            \end{minipage}
        \end{multicols}

        % Contradiction Rule
        \noindent
        \begin{minipage}{\linewidth}
            \[
                \infer[\mathcal{T}_0 \bot]{\bot}{
                    \begin{array}{ll}
                        A^\alpha & \\
                        \vdots & \\
                        A^\beta & (\alpha \neq \beta)
                    \end{array}
                }
            \]
        \end{minipage}

        \par\vspace{0.5em}
        \noindent
        \begin{tikzpicture}
            \draw[dashed] (0,0) -- (\linewidth,0);
        \end{tikzpicture}
        \par\vspace{0.5em}

        \begin{multicols}{3}
            \noindent
            \begin{minipage}{\linewidth}
                \[
                    \infer[]{A^F \mid B^T}{(A \Rightarrow B)^T}
                \]
            \end{minipage}

            \noindent
            \begin{minipage}{\linewidth}
                \[
                    \infer[]{
                        \begin{array}{c}
                            A^T \\
                            B^F
                        \end{array}
                    }{(A \Rightarrow B)^F}
                \]
            \end{minipage}

            \noindent
            \begin{minipage}{\linewidth}
                \vspace{-0.4cm}
                \[
                    \infer[]{B^T}{
                        \begin{array}{c}
                            A^T \\
                            (A \Rightarrow B)^T
                        \end{array}
                    }
                \]
            \end{minipage}
        \end{multicols}

        \begin{multicols}{2}
            \noindent
            \begin{minipage}{\linewidth}
                \[
                    \infer[]{A^T \mid B^T}{(A \lor B)^T}
                \]
            \end{minipage}

            \noindent
            \begin{minipage}{\linewidth}
                \[
                    \infer[]{
                        \begin{array}{c}
                            A^F \\
                            B^F
                        \end{array}
                    }{(A \lor B)^F}
                \]
            \end{minipage}
        \end{multicols}

        \begin{multicols}{2}
            \noindent
            \begin{minipage}{\linewidth}
                \[
                    \infer[]{
                        \begin{array}{c} A^T \\ B^T \end{array}
                        \mid
                        \begin{array}{c} A^F \\ B^F \end{array}
                    }{(A \iff B)^T}
                \]
            \end{minipage}

            \noindent
            \begin{minipage}{\linewidth}
                \[
                    \infer[]{
                        \begin{array}{c} A^T \\ B^F \end{array}
                        \mid
                        \begin{array}{c} A^F \\ B^T \end{array}
                    }{(A \iff B)^F}
                \]
            \end{minipage}
        \end{multicols}
    \end{tcolorbox}
    \caption{Propositional Tableau Calculus $\mathcal{T}_0$}
    \label{fig:t0_calculus}
\end{figure}


\definition{\linkterm{$\mathcal{T}_0$}{t0_calculus}-Theorem}{
$A$ is a \linkterm{$\mathcal{T}_0$}{t0_calculus}-theorem ($\vdash_{\linkterm{\mathcal{T}_0}{t0_calculus}} A$), iff there is a \linkterm{closed tableau}{closed_tableau} with $A^F$ at the \linkterm{root}{tree}.
}{t0_theorem}

\definition{}{
A \linkterm{labeled formula}{labeled_formula} $A^{\alpha}$ is \textbf{valid under} $\varphi$, iff $\mathcal{I}_{\varphi}(A) = \alpha$
}{valid_under_labeled}

\definition{Satisfiable Tableau}{
A tableau $\mathcal{T}$ is \textbf{satisfiable}, iff there is a satisfiable branch $\mathcal{B}$ in $\mathcal{T}$, i.e. if the \linkterm{set}{def:set} of \linkterm{formulae}{wff0_def} on $\mathcal{B}$ is \linkterm{satisfiable}{satisfiable_ls}.
}{satisfiable_tableau}

\Theorem{
\linkterm{$\mathcal{T}_0$}{t0_calculus} is \linkterm{sound}{sound}, i.e. $\Phi \subseteq \text{wff}_0(\Sigma_{\text{PL}^0}, \mathcal{V}_{\text{PL}^0})$ is \linkterm{valid}{valid_ls} if there is a \linkterm{closed tableau}{closed_tableau} $\mathcal{T}$ for $\Phi^F$.    
}{thm:soundT0}

\Theorem{
\linkterm{$\mathcal{T}_0$}{t0_calculus} is \linkterm{complete}{complete}, i.e. if $\Phi \subseteq \text{wff}_0(\Sigma_{\text{PL}^0}, \mathcal{V}_{\text{PL}^0})$ is \linkterm{valid}{valid_ls}, then there is a \linkterm{closed tableau}{closed_tableau} $\mathcal{T}$ for $\Phi^F$.    
}{thm:completeT0}

\Lemma{
\linkterm{$\mathcal{T}_0$}{t0_calculus} terminates, i.e. every \linkterm{$\mathcal{T}_0$ tableau}{t0_calculus} becomes \linkterm{saturated}{saturated_tableau} after \linkterm{finitely}{set_cardinality} many \linkterm{rule applications}{inference_rules_logic}.
}{lemma_t0_terminates}

\Corollary{
\linkterm{$\mathcal{T}_0$}{t0_calculus} induces a decision procedure for \linkterm{validity}{valid_ls} in \linkterm{$\text{PL}^0$}{def:pl0_as_ls}.    
}{cor:decision_T0}
\Proof{
\begin{itemize}
    \item By \Lemref{lemma_t0_terminates}, it is decidable whether $\vdash_{\linkterm{\mathcal{T}_0}{t0_calculus}} A$
    \item By \Thmref{thm:soundT0} and \Thmref{thm:completeT0}, $\vdash_{\linkterm{\mathcal{T}_0}{t0_calculus}} A$ iff $A$ is \linkterm{valid}{valid_ls}
\end{itemize}    
}

\textbf{Example: } \Figref{t0proofexample} shows a \linkterm{$\mathcal{T}_0$}{t0_calculus} proof of the formula $(A \lor B) \land (A \Rightarrow C) \land (B \Rightarrow C) \Rightarrow C$


\begin{figure}[H]
    \centering
    \begin{tcolorbox}[colback=white, colframe=black, sharp corners, boxrule=0.5pt]
        \centering
        \begin{tikzpicture}[node distance=1ex]
            \node (F) {$((A \lor B) \land (A \Rightarrow C) \land (B \Rightarrow C) \Rightarrow C)^F$};
            \node (F1) [below=of F] {$((A \lor B) \land (A \Rightarrow C) \land (B \Rightarrow C))^T$};
            \node (F2) [below=of F1] {$C^F$};
            \node (AvB) [below=of F2] {$(A \lor B)^T$};
            \node (AtoC) [below=of AvB] {$(A \Rightarrow C)^T$};
            \node (BtoC) [below=of AtoC] {$(B \Rightarrow C)^T$};
            
            % First Branch
            \node (B) [below left=of BtoC] {$B^F$};
            \node (C) [below right=of BtoC] {$C^T$};
            \node (bot1) [below=of C] {$\bot$};
            
            % Second Branch (from B)
            \node (AF) [below left=of B] {$A^F$};
            \node (CT) [below right=of B] {$C^T$};
            \node (bot2) [below=of CT] {$\bot$};
            
            % Third Branch (from AF)
            \node (AT) [below left=of AF] {$A^T$};
            \node (bot3) [below=of AT] {$\bot$};
            \node (BT) [below right=of AF] {$B^T$};
            \node (bot4) [below=of BT] {$\bot$};
        
            % Edges
            \path (BtoC) edge[-] (B);
            \path (BtoC) edge[-] (C);
        
            \path (B) edge[-] (AF);
            \path (B) edge[-] (CT);
        
            \path (AF) edge[-] (AT);
            \path (AF) edge[-] (BT);
        \end{tikzpicture}
    \end{tcolorbox}
    \caption{\linkterm{$\mathcal{T}_0$}{t0_calculus} Proof Example}
    \label{t0proofexample}
\end{figure}