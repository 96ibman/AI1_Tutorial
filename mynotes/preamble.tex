\usepackage[
backend=biber,
style=alphabetic,
sorting=ynt
]{biblatex}
\addbibresource{refs.bib}

% ===================================================================
% CORE DOCUMENT UTILITIES
% ===================================================================
\usepackage[margin=0.5in]{geometry}   % Page geometry
\usepackage{graphicx}                 % Images
\usepackage{float}                    % Force float placement [H]
\usepackage{booktabs}                 % Professional table rules
\usepackage{array}                    % Advanced table column definitions
\usepackage{multicol}                 % Multi-column layout
\usepackage{parskip}                  % No paragraph indent, vertical spacing
\usepackage{titling}                  % Control title spacing
\setlength{\droptitle}{-1cm}         % Optional: pull title upward

% ===================================================================
% COLORS, TEXT, AND LISTS
% ===================================================================
\usepackage{xcolor, soul}             % Color + highlighting
\usepackage[inline]{enumitem}         % Inline and compact lists
\usepackage{gensymb}                  % Symbols like \degree
\usepackage{xspace}                   % Smart spacing after macros

% ===================================================================
% MATHEMATICS, LOGIC, AND SEMANTICS
% ===================================================================
\usepackage{amsmath, amssymb, mathtools} % Math foundations
\usepackage{mathrsfs}                % Script fonts (\mathscr)
\usepackage{stmaryrd}                % Semantic brackets ⟦ ⟧
\usepackage{mathpartir}              % Inference rules
\usepackage{semantic}                % Semantics macros

% --- Paired delimiters ---
\newcommand{\set}[1]{\left\{#1\right\}}
\newcommand{\abs}[1]{\left\lvert#1\right\rvert}
\newcommand{\norm}[1]{\left\lVert#1\right\rVert}
\newcommand{\paren}[1]{\left( #1 \right)}
\newcommand{\brak}[1]{[#1]}

% --- Common sets ---
\providecommand{\N}{\mathbb{N}}
\providecommand{\Z}{\mathbb{Z}}
\providecommand{\Q}{\mathbb{Q}}
\providecommand{\R}{\mathbb{R}}

% --- Math operators ---
\newcommand{\argmax}[1]{\mathop{\mathrm{arg\,max}}_{#1}\,}
\newcommand{\argmin}[1]{\mathop{\mathrm{arg\,min}}_{#1}\,}

% --- Set and tuple utilities ---
\newcommand{\tuple}[1]{\left\langle #1 \right\rangle}
\newcommand{\restr}[2]{#1\!\upharpoonright_{#2}} % Restriction f|_A
\newcommand{\sint}[1]{\llbracket #1 \rrbracket}  % Semantic brackets
\newcommand{\setsize}[1]{\Gamma (#1)}            % Size-related macro
\newcommand{\inverse}[1]{\ensuremath{{#1}^{-1}}} % Inverse
\newcommand{\powerset}[1]{\ensuremath{\mathcal{P}(#1)}} % Powerset
\newcommand{\setbuilder}[2]{\ensuremath{\{\,#1 \mid #2\,\}}} % Set-builder

% --- LaTeX3 function syntax macros ---
\ExplSyntaxOn
\NewDocumentCommand{\union}{m}{\ensuremath{\clist_use:nn {#1} { \mathbin{\cup} }}}
\NewDocumentCommand{\intersection}{m}{\ensuremath{\clist_use:nn {#1} { \mathbin{\cap} }}}
\NewDocumentCommand{\compose}{m}{\ensuremath{\clist_use:nn {#1} { \mathbin{\circ} }}}
\NewDocumentCommand{\cartprod}{m}{\ensuremath{\clist_use:nn {#1} { \mathbin{\times} }}}
\ExplSyntaxOff

% --- Total and Partial Functions ---
\newcommand{\func}[3]{\ensuremath{#1 : #2 \to #3}}
\newcommand{\pfunc}[3]{\ensuremath{#1 : #2 \rightharpoonup #3}}

% --- Indexed set operations ---
\newcommand{\bigunion}[1]{\ensuremath{\bigcup_{i \in I} {#1}_{i}}}
\newcommand{\bigintersection}[1]{\ensuremath{\bigcap_{i \in I} {#1}_{i}}}

% --- Common math helpers ---
\newcommand{\identity}[1]{\ensuremath{\operatorname{Id}_{#1}}}
\newcommand{\bigo}[1]{\mathcal{O}(#1)}

% ===================================================================
% TIKZ AND DIAGRAMS
% ===================================================================
\usepackage{tikz}
\usetikzlibrary{arrows.meta, positioning}
\tikzset{
  vertex/.style={circle, draw, minimum size=18pt, inner sep=0pt},
  edge/.style={-},
  arc/.style={->, >=Stealth}
}

% ===================================================================
% CODE, ALGORITHMS, AND BOXES
% ===================================================================
\usepackage{tcolorbox}
\tcbuselibrary{listingsutf8}

\usepackage{algorithm, algorithmicx, algpseudocode} % Algorithm support

\usepackage[utf8]{inputenc}
\usepackage{textcomp}

% --- Note boxes ---
\newcommand{\commandnote}[1]{%
  \begin{tcolorbox}[standard jigsaw, title=Note]
    #1
  \end{tcolorbox}
}

% --- Highlighted inline text ---
\newcommand{\hltext}[1]{\sethlcolor{gray!10}\hl{#1}}

% --- Code block environment ---
\newtcblisting{codeblock}{
  colback=gray!10,
  colframe=black!40,
  listing only,
  boxrule=0.5pt,
  arc=2pt,
  left=6pt, right=6pt, top=4pt, bottom=4pt,
  listing options={
    breaklines=true,
    basicstyle=\ttfamily,
    literate={\$}{{\textdollar}}1
  }
}

% --- Inline code snippet ---
\newcommand{\inlinecode}[1]{\colorbox{gray!10}{\texttt{#1}}}

% ===================================================================
% BIBLIOGRAPHY AND HYPERLINKS
% ===================================================================
\usepackage{hyperref}                   % Hyperlinks
\hypersetup{
  colorlinks,
  citecolor=teal,
  linkcolor=blue,
  urlcolor=magenta
}
\usepackage[nameinlink,capitalize]{cleveref} % Smart cross-references
% ===================================================================
% CUSTOM STRUCTURES AND LABELS
% ===================================================================
% --- Section titles ---
\newcommand{\labeledsection}[2]{\section{#1}\label{#2}}
\newcommand{\minititle}[1]{\subsubsection*{#1}}

% --- Anchors and references ---
\newcommand{\refterm}[2]{\phantomsection \label{#2}{\textbf{#1}}}
\newcommand{\linkterm}[2]{\hyperref[#2]{#1}}

% ===================================================================
% DEFINITIONS AND THEOREMS
% ===================================================================
% --- Section-scoped definitions ---
\newcounter{definition}[section]
\renewcommand{\thedefinition}{\thesection.\arabic{definition}}
\newcommand{\definition}[3]{%
  \refstepcounter{definition}%
  \phantomsection%
  \par\noindent\textbf{Def~\thedefinition.} \textbf{#1: }#2%
  \label{#3}\par
}

% --- Subsection-scoped definitions ---
\newcounter{subdefinition}[subsection]
\renewcommand{\thesubdefinition}{\thesubsection.\arabic{subdefinition}}
\newcommand{\subdefinition}[3]{%
  \refstepcounter{subdefinition}%
  \phantomsection%
  \par\noindent\textbf{Definition~\thesubdefinition.} \textbf{#1: }#2%
  \label{#3}\par
}

% --- Shared theorem counter per section ---
\newcounter{statement}[section]
\renewcommand{\thestatement}{\thesection.\arabic{statement}}

% --- Linked theorem-type counters ---
\newcounter{theorem}
\renewcommand{\thetheorem}{\thestatement}
\newcounter{lemma}
\renewcommand{\thelemma}{\thestatement}
\newcounter{corollary}
\renewcommand{\thecorollary}{\thestatement}

% --- Statement macros ---
\newcommand{\Theorem}[2]{%
  \stepcounter{statement}\refstepcounter{theorem}%
  \phantomsection%
  \par\noindent\textbf{Theorem~\thetheorem.} #1%
  \label{#2}\par
}
\newcommand{\Lemma}[2]{%
  \stepcounter{statement}\refstepcounter{lemma}%
  \phantomsection%
  \par\noindent\textbf{Lemma~\thelemma.} #1%
  \label{#2}\par
}
\newcommand{\Corollary}[2]{%
  \stepcounter{statement}\refstepcounter{corollary}%
  \phantomsection%
  \par\noindent\textbf{Corollary~\thecorollary.} #1%
  \label{#2}\par
}
\newcommand{\Proof}[1]{%
  \par\noindent\textit{Proof.} #1\hfill$\square$\par
}

% ===================================================================
% CLEVEREF CONFIGURATION
% ===================================================================
\crefname{section}{section}{sections}
\Crefname{section}{Section}{Sections}
\crefname{subsection}{section}{sections}
\Crefname{subsection}{Section}{Sections}
\crefname{figure}{figure}{figures}
\Crefname{figure}{Figure}{Figures}
\crefname{table}{table}{tables}
\Crefname{table}{Table}{Tables}
\crefname{equation}{equation}{equations}
\Crefname{equation}{Equation}{Equations}
\crefname{algorithm}{algorithm}{algorithms}
\Crefname{algorithm}{Algorithm}{Algorithms}
\crefname{theorem}{theorem}{theorems}
\Crefname{theorem}{Theorem}{Theorems}
\crefname{lemma}{lemma}{lemmas}
\Crefname{lemma}{Lemma}{Lemmas}
\crefname{corollary}{corollary}{corollaries}
\Crefname{corollary}{Corollary}{Corollaries}
\crefname{appendix}{appendix}{appendices}
\Crefname{appendix}{Appendix}{Appendices}
\Crefname{definition}{Def}{Defs}
\Crefname{subdefinition}{Def}{Defs}

% --- Reference shorthands ---
\newcommand{\secref}[1]{\cref{#1}}   \newcommand{\Secref}[1]{\Cref{#1}}
\newcommand{\figref}[1]{\cref{#1}}   \newcommand{\Figref}[1]{\Cref{#1}}
\newcommand{\tabref}[1]{\cref{#1}}   \newcommand{\Tabref}[1]{\Cref{#1}}
\newcommand{\eqrefc}[1]{\cref{#1}}   \newcommand{\Eqref}[1]{\Cref{#1}}
\newcommand{\defref}[1]{\cref{#1}}   \newcommand{\Defref}[1]{\Cref{#1}}
\newcommand{\Subdefref}[1]{\Cref{#1}}
\newcommand{\thmref}[1]{\cref{#1}}   \newcommand{\Thmref}[1]{\Cref{#1}}
\newcommand{\lemref}[1]{\cref{#1}}   \newcommand{\Lemref}[1]{\Cref{#1}}
\newcommand{\corref}[1]{\cref{#1}}   \newcommand{\Corref}[1]{\Cref{#1}}
\newcommand{\appref}[1]{\cref{#1}}   \newcommand{\Appref}[1]{\Cref{#1}}
\newcommand{\Algref}[1]{\Cref{#1}}