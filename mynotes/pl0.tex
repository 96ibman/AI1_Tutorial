\labeledsection{Propositional Logic}{sec:pl0}
\definition{Propositional Logic}{
Propositional Logic ($\text{PL}^0$) is a \linkterm{logical system}{logical_system} $\tuple{\mathcal{L},\mathcal{M},\vDash}$ Where
\begin{itemize}
    \item the \linkterm{formal language}{formal_language} $\mathcal{L}$ is defined as the well-formed formulae of propositional logic: $\text{wff}_0(\Sigma_{\text{PL}^0}, \mathcal{V}_{\text{PL}^0})$,
    \item the \linkterm{set}{def:set} $\mathcal{M}$ of models is given by the \linkterm{set}{def:set} $\mathcal{K}_0$ of \linkterm{total functions}{function} $\func{\varphi}{\mathcal{V}_0}{\mathcal{D}_0}$ (i.e., propositional variable \linkterm{assignments}{var_ass_csp}), 
    \item and $\varphi \vDash A$ iff $\mathcal{I}_{\varphi}(A) = T$ (a \linkterm{proposition}{proposition} is satisfiable by model $\varphi$ if the interpretation of it -under that $\varphi$- is true)
\end{itemize}
Hence, propositional logic is a \linkterm{logical system}{logical_system} $\tuple{\text{wff}_0(\Sigma_{\text{PL}^0}, \mathcal{V}_{\text{PL}^0}),\mathcal{K}_0,\vDash}$
}{def:pl0_as_ls}

\definition{Formulae of \linkterm{$\text{PL}^0$}{def:pl0_as_ls}}{
The \linkterm{formulae}{formulae} of \linkterm{$\text{PL}^0$}{def:pl0_as_ls} are made up from 
\begin{itemize}
    \item propositional variables $\mathcal{V}_{\text{PL}^0}$ (or just $\mathcal{V}_0$) : $\mathcal{V}_0 := \set{P,Q,R, \cdots}$ (\linkterm{countably infinite}{countably_infinite})
    \item propositional signature $\Sigma_{\text{PL}^0}$ (or just $\Sigma_0$): $\Sigma_0 := \set{\top,\bot,\neg, \lor, \land, \Rightarrow, \Leftrightarrow, \cdots}$ (called \textit{connectives})
\end{itemize}
We define the \linkterm{set}{def:set} $\text{wff}_0(\Sigma_{\text{PL}^0}, \mathcal{V}_{\text{PL}^0})$ of \textbf{well-formed propositional formulae (wffs)} as:
\begin{itemize}
    \item propositional variables
    \item logical constants $\top$ and $\bot$
    \item negations $\neg A$
    \item conjunctions $A \land B$
    \item disjunctions $A \lor B$
    \item implications $A \Rightarrow B$, and 
    \item equivalence (biimplications) $A \Leftrightarrow B$
\end{itemize}
where $A,B \in \text{wff}_0(\Sigma_{\text{PL}^0}, \mathcal{V}_{\text{PL}^0})$ themselves.
}{wff0_def}

\definition{Grammar for \linkterm{$\text{PL}^0$}{def:pl0_as_ls}}{
The \linkterm{set}{def:set} of \linkterm{propositional formulae}{wff0_def} $\set{A \mid A \in \text{wff}_0(\Sigma_{\text{PL}^0}, \mathcal{V}_{\text{PL}^0})}$ is given by the following \linkterm{grammar}{phrase_structure_grammar}:
\begin{align*}
X \quad &{::=} \quad  P \mid Q \mid R \mid \cdots \quad (\textit{propositional variables}) \\
A \quad &{::=} \quad X \mid \top \mid \bot \mid \neg A \mid A \land A \mid A \lor A \mid A \Rightarrow A \mid A \Leftrightarrow A
\end{align*}
}{wff0_grammar}

\definition{Canonical Model}{
We call a \linkterm{logical system}{logical_system} $\tuple{\mathcal{L},\mathcal{M},\vDash}$ a \textbf{single-model} with \textbf{canonical model} $M$ iff $\mathcal{M} = \set{M}$
}{canonical_model}

\definition{\linkterm{$\text{PL}^0$}{def:pl0_as_ls} Models}{
A model $\mathcal{M} := \tuple{\mathcal{D}_0, \mathcal{I}}$ for \linkterm{$\text{PL}^0$}{def:pl0_as_ls} consists of 
\begin{itemize}
    \item the universe $\mathcal{D}_0 := \set{T,F}$ (truth values)
    \item the interpretation $\mathcal{I}$ that \linkterm{assigns}{var_ass_csp} truth values to essential connectives
    \begin{enumerate}
        \item $\func{\mathcal{I}(\neg)}{\mathcal{D}_0}{\mathcal{D}_0}$; $T \mapsto F, F \mapsto T$
        \item $\func{\mathcal{I}(\land)}{\cartprod{\mathcal{D}_0,\mathcal{D}_0}}{\mathcal{D}_0}$; $\tuple{\alpha, \beta} \mapsto T$, iff $\alpha = \beta = T$
        \item $\mathcal{I}(\top) = T$
        \item $\mathcal{I}(\bot) = F$
    \end{enumerate}
\end{itemize}
}{model_pl0}

\commandnote{
\linkterm{$\text{PL}^0$}{def:pl0_as_ls} is a single-model \linkterm{logical system}{logical_system} with \linkterm{canonical model}{canonical_model} $\tuple{\mathcal{D}_0, \mathcal{I}}$
}

\definition{Variable Assignment}{
\linkterm{$\text{PL}^0$}{def:pl0_as_ls} uses a \linkterm{total}{function} \linkterm{variable assignment}{var_ass_csp} $\func{\varphi}{\mathcal{V}_0}{\mathcal{D}_0}$ that assigns truth values to propositional variables.
}{var_ass_pl0}

\definition{Value Function}{
\linkterm{$\text{PL}^0$}{def:pl0_as_ls} uses a value \linkterm{function}{function} $\func{\mathcal{I}_{\varphi}}{\text{wff}_0(\Sigma_{\text{PL}^0}, \mathcal{V}_{\text{PL}^0})}{\mathcal{D}_0}$ that assigns truth values to \linkterm{$\text{PL}^0$}{def:pl0_as_ls} \linkterm{formulae}{wff0_def}. It is defined as follows:
\begin{itemize}
\item  $\mathcal{I}_{\varphi}(P) = \varphi(P)$
\item $\mathcal{I}_{\varphi}(\neg A) = \mathcal{I}(\neg)(\mathcal{I}_{\varphi}(A))$
\item $\mathcal{I}_{\varphi}(A \land B) = \mathcal{I}(\land)(\mathcal{I}_{\varphi}(A),\mathcal{I}_{\varphi}(B))$
\end{itemize}
}{value_function_pl0}

\commandnote{
\textbf{Value Function Example:} Compute $\mathcal{I}_\varphi(P \lor Q)$ under the assignment $\varphi := \set{P \mapsto F, Q \mapsto T}$
\begin{align*}
    \mathcal{I}_\varphi(P \lor Q) &= \mathcal{I}_\varphi(\neg(\neg P \land \neg Q)) \\
    \mathcal{I}_\varphi(\neg P) &= \mathcal{I}(\neg)(\mathcal{I}_\varphi(P)) \\
    \mathcal{I}_\varphi(\neg Q) &= \mathcal{I}(\neg)(\mathcal{I}_\varphi(Q)) \\
    \mathcal{I}_\varphi(\neg P \land \neg Q) &= \mathcal{I}(\land)(\mathcal{I}_\varphi(\neg P),\mathcal{I}_\varphi(\neg Q))\\
    \mathcal{I}_\varphi(P \lor Q) &= \mathcal{I}(\neg)(\mathcal{I}_\varphi(\neg P \land \neg Q))\\
    &= \mathcal{I}(\neg)(\mathcal{I}(\land)(\mathcal{I}_\varphi(\neg P),\mathcal{I}_\varphi(\neg Q)))\\
    &= \mathcal{I}(\neg)(\mathcal{I}(\land)
    (\mathcal{I}(\neg)(\mathcal{I}_\varphi(P)),
    \mathcal{I}(\neg)(\mathcal{I}_\varphi(Q))))
\end{align*}
We know that
\begin{itemize}
    \item $\mathcal{I}_{\varphi}(P) = \varphi(P) = F$, and
    \item $\mathcal{I}_{\varphi}(Q) = \varphi(Q) = T$
\end{itemize}
Hence:
\begin{align*}
&= \mathcal{I}(\neg)(\mathcal{I}(\land)(\mathcal{I}(\neg)(F),\mathcal{I}(\neg)(T))) \\ 
&= \mathcal{I}(\neg)(\mathcal{I}(\land)(T,F)) \\
&= \mathcal{I}(\neg)(F) \\
&= T  
\end{align*}
}

\definition{Variable Occurrence}{
A \linkterm{function}{function} that maps a \linkterm{$\text{PL}^0$ formula}{wff0_def} to the \linkterm{set}{def:set} of variable occurred in that formula. It is defined as:
\begin{itemize}
    \item $\text{Var}(P) = \set{P}$
    \item $\text{Var}(\top) = \emptyset$
    \item $\text{Var}(\bot) = \emptyset$
    \item $\text{Var}(\neg A) = \text{Var}(A)$
    \item $\text{Var}(A \circ B) = \text{Var}(A) \cup \text{Var}(B)$ where $\circ \in \set{\land, \lor, \Rightarrow, \Leftrightarrow}$
\end{itemize}
}{var_occ_pl0}

\definition{Ground Formula}{
A \linkterm{$\text{PL}^0$}{def:pl0_as_ls} formula $A$ is called \textbf{ground} iff \linkterm{Var}{var_occ_pl0}$(A) = \emptyset$
}{ground_pl0}

\commandnote{
\textbf{Alternative Notation: }write $\sint{A}_{\varphi}$ for $\mathcal{I}_{\varphi}(A)$ (and $\sint{A}$ if $A$ is \linkterm{ground}{ground_pl0})
}

\definition{Equivalent}{
Two \linkterm{formulae}{wff0_def} $A$ and $B$ are called \textbf{equivalent}, iff $\mathcal{I}_{\varphi}(A) = \mathcal{I}_{\varphi}(B)$ for all \linkterm{variable assignments}{var_ass_pl0} $\varphi$
}{equivalent_formulae}

\definition{\linkterm{$\text{PL}^0$}{def:pl0_as_ls} Identities}{
\Tabref{tab:identities_pl0} shows the identities in \linkterm{$\text{PL}^0$}{def:pl0_as_ls}
}{identities_pl0}

\begin{table}[h]
\centering
\caption{\linkterm{$\text{PL}^0$}{def:pl0_as_ls} Identities}
\label{tab:identities_pl0}
\begin{tabular}{|l|c|c|}
\hline
\textbf{Name} & \textbf{for $\land$} & \textbf{for $\lor$} \\
\hline
Idempotence      & $\varphi \land \varphi = \varphi$        & $\varphi \lor \varphi = \varphi$ \\
Identity         & $\varphi \land \top = \varphi$            & $\varphi \lor \bot = \varphi$ \\
Absorption 1     & $\varphi \land \bot = \bot$                & $\varphi \lor \top = \top$ \\
Commutativity    & $\varphi \land \psi = \psi \land \varphi$ & $\varphi \lor \psi = \psi \lor \varphi$ \\
Associativity    & $\varphi \land (\psi \land \theta) = (\varphi \land \psi)\land \theta$  
                 & $\varphi \lor (\psi \lor \theta) = (\varphi \lor \psi)\lor \theta$ \\
Distributivity   & $\varphi \land (\psi \lor \theta)= (\varphi \land \psi)\lor (\varphi \land \theta)$ 
                 & $\varphi \lor (\psi \land \theta)= (\varphi \lor \psi)\land (\varphi \lor \theta)$ \\
Absorption 2     & $\varphi \land (\varphi \lor \theta)= \varphi$ 
                 & $\varphi \lor (\varphi \land \theta)= \varphi$ \\
De Morgan rule   & $\neg(\varphi \land \psi)= \neg\varphi \lor \neg\psi$ 
                 & $\neg(\varphi \lor \psi)= \neg\varphi \land \neg\psi$ \\
double negation  & \multicolumn{2}{c|}{$\neg\neg\varphi = \varphi$} \\
\hline
Definitions      & $\varphi \Rightarrow \psi \;=\; \neg\varphi \lor \psi$ 
                 & $\varphi \Leftrightarrow \psi \;=\; (\varphi \Rightarrow \psi)\land(\psi \Rightarrow \varphi)$ \\
\hline
\end{tabular}
\end{table}

Let $\M := \tuple{\D_0, \I}$ be our \linkterm{model}{model_pl0}, then we say that a \linkterm{formula}{wff0_def} $A$ is 
\begin{itemize}
\item \refterm{true under}{true_under_pl0} $\varphi$ in $\M$, iff $\Ivof{A} = T$, (write $\M \satphi A$) 
\item \refterm{falsifies}{falsifies_pl0} $\varphi$ in $\M$, iff $\Ivof{A} = F$, (write $\M \unsatphi A$)
\item \refterm{satisfiable in}{sat_in_pl0} $\M$, iff $\Ivof{A} = T$ for some \linkterm{assignment}{var_ass_pl0} $\varphi$
\item \refterm{valid in}{valid_in_pl0} $M$, iff $\M \satphi A$ for all \linkterm{variable assignments}{var_ass_pl0} $\varphi$
\item \refterm{falsifiable in}{falsifiable_in_pl0} $\M$, iff $\Ivof{A} = F$ for some \linkterm{assignment}{var_ass_pl0} $\varphi$, and
\item \refterm{unsatisfiable in}{unsat_in_pl0} $\M$, iff $\Ivof{A} = F$ for all \linkterm{variable assignments}{var_ass_pl0} $\varphi$. 
\end{itemize}

\commandnote{
Since \linkterm{$\text{PL}^0$}{def:pl0_as_ls} is a \linkterm{single-model}{canonical_model} \linkterm{logical system}{logical_system}, we can omit the explicit reference to the \linkterm{canonical model}{canonical_model} and write $\varphi \sat A$ to mean "$A$ is true under $\varphi$", recovering the notation from $\M \satphi A$.
}

\definition{Entailment}{
We say that $A$ \textbf{entails} $B$ (write $A \sat B$), iff $\Ivof{B} = T$ for all $\varphi$ with $\Ivof{A} = T$ (i.e. all \linkterm{variable assignments}{var_ass_pl0}) that makes $A$ true also make $B$ true
}{entailment_pl0}

\Theorem{
$A \sat B \Rightarrow A \land C \sat B \land C$
}{thm:entailment}
\Proof{
\begin{enumerate}
\item Assume that $A \sat B$ ($\mathcal{H}_1$), we want to show that for all $\varphi$ that makes $A \land C$ true also make $B \land C$ true 
\item Assume there is some $\varphi$ that makes $A \land C$ true $\leadsto$ $\Ivof{A \land C} = T$ ($\mathcal{H}_2$)
\item This means that $\Ivof{A} = \Ivof{C} = T$
\item Given $\mathcal{H}_1$ we now know that our $\varphi$ also make $B$ true $\leadsto \Ivof{B}=T$
\item From 3 and 4 and by the definition of conjunction: $\Ivof{B \land C}=T$
\end{enumerate}
}

\definition{Soundness}{
Let $\mathcal{S}:=\tuple{\mathcal{L},\mathcal{M},\vDash}$ be a \linkterm{logical system}{logical_system} and $\mathcal{H} \subseteq \mathcal{L}$ a \linkterm{set}{def:set} of hypotheses, then we call a \linkterm{calculus}{calculus_logic} $\mathcal{C}$ over $\mathcal{L}$ \textbf{sound} (correct), iff $\mathcal{H} \sat A$ whenever $\mathcal{H} \vdash_{\mathcal{C}} A$
}{sound}

\definition{Completeness}{
Let $\mathcal{S}:=\tuple{\mathcal{L},\mathcal{M},\vDash}$ be a \linkterm{logical system}{logical_system} and $\mathcal{H} \subseteq \mathcal{L}$ a \linkterm{set}{def:set} of hypotheses, then we call a \linkterm{calculus}{calculus_logic} $\mathcal{C}$ over $\mathcal{L}$ \textbf{complete}, iff $\mathcal{H} \vdash_{\mathcal{C}} A$ whenever $\mathcal{H} \sat A$ 
}{complete}

\definition{Hilbert Calculus}{
The Hilbert calculus $\mathcal{H}^0$ is a \linkterm{calculus}{calculus_logic} that consists of the following \linkterm{inference rules}{inference_rules_logic}:

\[
    \infer[K]{P \Rightarrow (Q \Rightarrow P)}{}
    \qquad
    \infer[S]{(P \Rightarrow (Q \Rightarrow R)) \Rightarrow ((P \Rightarrow Q) \Rightarrow (P \Rightarrow R))}{}
\]

\[
    \infer[\text{MP}]
    {B}
    {A \Rightarrow B \quad A}
    \qquad
    \infer[\text{Subst}]
    {[B/X](A)}
    {A}
\]
}{hilbert_calculus}

\commandnote{
Subst \linkterm{inference rule}{inference_rules_logic} is \linkterm{admissible}{admissible_inference_rule} but not \linkterm{derivable}{derived_inference_rule}
}

\definition{Hilbert Formal System}{
Using \linkterm{propositional logic}{def:pl0_as_ls} and the \linkterm{Hilbert Calculus}{hilbert_calculus}, we can build a simple \linkterm{formal system}{formal_system}:
\[
\langle \underbrace{\langle \mathcal{L}, \mathcal{M}, \vDash \rangle}_{\text{\linkterm{propositional logic}{def:pl0_as_ls}}}, \underbrace{\mathcal{H}^0}_{\text{\linkterm{Hilbert Calculus}{hilbert_calculus}}} \rangle
\]
}{hilbert_formal_system}

\textbf{Example: }$C \Rightarrow C$ is a $\mathcal{H}^0$ and here is its \linkterm{proof}{proofs}:

\Proof{
We show that $\emptyset \vdash_{\mathcal{H}^0} C \Rightarrow C$
\begin{enumerate}
\item \linkterm{axiom}{axiom_logic} $S$ gives us:
\[
(P \Rightarrow (Q \Rightarrow R)) \Rightarrow ((P \Rightarrow Q) \Rightarrow (P \Rightarrow R))
\] 
\item We apply Subst with $[C/P], [C \Rightarrow C / Q], [C / R]$, we get:
\[
(C \Rightarrow ((C \Rightarrow C) \Rightarrow C)) \Rightarrow ((C \Rightarrow (C \Rightarrow C)) \Rightarrow (C \Rightarrow C))
\] 
\item \linkterm{axiom}{axiom_logic} $K$ gives us:
\[
P \Rightarrow (Q \Rightarrow P)
\]
\item We apply Subst with $[C / P], [C \Rightarrow C / Q]$, we get:
\[
C \Rightarrow ((C \Rightarrow C) \Rightarrow C)
\]
\item We use MP on 4 and 2, we get:
\[
(C \Rightarrow (C \Rightarrow C)) \Rightarrow (C \Rightarrow C)
\]
\item \linkterm{axiom}{axiom_logic} $K$ gives us:
\[
P \Rightarrow (Q \Rightarrow P)
\]
\item We apply Subst with $[C / P], [C / Q]$, we get:
\[
C \Rightarrow (C \Rightarrow C)
\]
\item We apply MP on 7 and 5, we get:
\[
C \Rightarrow C
\]
\end{enumerate}
}

\definition{$\mathcal{ND}_0$}{
The \textbf{propositional natural deduction calculus} $\mathcal{ND}_0$ has \linkterm{inference rules}{inference_rules_logic} for the introduction and elimination of connectives. $\mathcal{ND}_0$ \linkterm{rules}{inference_rules_logic} are shown in \figref{fig:nd0_rules}
}{nd_0}

\begin{figure}[H]
    \centering
    \begin{tcolorbox}[colback=white, colframe=black, sharp corners, boxrule=0.5pt]
        
        \begin{multicols}{3}
            % conj. intro
            \noindent
            \begin{minipage}{\linewidth}   
                \[
                    \infer[\land I]{A \land B}{A & B}
                \]
            \end{minipage}
        
            % conj. elimL
            \noindent
            \begin{minipage}{\linewidth}   
                \[
                    \infer[\land E_l]{A}{A \land B}
                \]
            \end{minipage}
        
            % conj. elimR
            \noindent
            \begin{minipage}{\linewidth}   
                \[
                    \infer[\land E_r]{B}{A \land B}
                \]
            \end{minipage}
        \end{multicols}
        
        \begin{multicols}{2}
            % impl. intro
            \noindent
            \begin{minipage}{\linewidth}
                \[
                    \infer[\Rightarrow I^1]{A \Rightarrow B}{\infer*{B}{[A]^1}}
                \]
            \end{minipage}
        
            % impl. elim.
            \noindent
            \begin{minipage}{\linewidth}
                \vspace*{1cm}
                \[
                    \infer[\Rightarrow E]{B}{A \Rightarrow B \quad A}
                \]
            \end{minipage}
        \end{multicols}
        
        \noindent
        \begin{minipage}{\linewidth}
            % TND
            $$
                \infer[\mathcal{TND}]{A \lor \neg A}{\quad}
            $$
        \end{minipage}
        
        \begin{multicols}{2}
            \noindent
            \begin{minipage}{\linewidth}
                \vspace{0.3cm}
                \[
                    \infer[\Leftrightarrow I]{A \Leftrightarrow A}{}
                \]
            \end{minipage}
        
            \noindent
            \begin{minipage}{\linewidth}
                \[
                    \infer[\Leftrightarrow E]{[B/p]C}{A \Leftrightarrow B \qquad C[A]_p}
                \]
            \end{minipage}
        \end{multicols}

        \par\vspace{0.5em}
        \noindent
        \begin{tikzpicture}
            \draw[dashed] (0,0) -- (\linewidth,0);
        \end{tikzpicture}
        \par\vspace{0.5em}
        
        \begin{multicols}{2}
            \noindent
            \begin{minipage}{\linewidth}
                \[
                    \infer[\lor I_l]{A \lor B}{A}
                \]
            \end{minipage}
        
            \noindent
            \begin{minipage}{\linewidth}
                \[
                    \infer[\lor I_r]{A \lor B}{B}
                \]
            \end{minipage}
        \end{multicols}
        
        \noindent
        \begin{minipage}{\linewidth}
            \[
                \infer[\lor E^1]{C}{
                    A \lor B \quad
                    \infer*{C}{[A]^1} \quad
                    \infer*{C}{[B]^1}
                }
            \]
        \end{minipage}
        
        \begin{multicols}{2}
            \noindent
            \begin{minipage}{\linewidth}
                \[
                    \infer[\neg I^1]{\neg A}{
                        \infer*{C}{[A]^1} \quad \infer*{\neg C}{[A]^1}
                    }
                \]
            \end{minipage}
        
            \noindent
            \begin{minipage}{\linewidth}
                \vspace*{1cm}
                \[
                    \infer[\neg E]{A}{\neg \neg A}
                \]
            \end{minipage}
        \end{multicols}
        
        \begin{multicols}{2}
            \noindent
            \begin{minipage}{\linewidth}
                \[
                    \infer[\bot I]{\bot}{\neg A \quad A}
                \]
            \end{minipage}
        
            \noindent
            \begin{minipage}{\linewidth}
                \[
                    \infer[\bot E]{A}{\bot}
                \]
            \end{minipage}
        \end{multicols}
        
    \end{tcolorbox}
    \caption{Propositional Natural Deduction Rules}
    \label{fig:nd0_rules}
\end{figure}

\commandnote{
\linkterm{Rules}{inference_rules_logic} below the dashed line "- - -" are all \linkterm{derived rules}{derived_inference_rule} from the original \linkterm{set}{def:set} above that line
}

\definition{$\mathcal{ND}_0$ Derivation Tree}{
Given a \linkterm{set}{def:set} $\mathcal{H} \subseteq \text{wff}_0(\Sigma_{\text{PL}^0}, \mathcal{V}_{\text{PL}^0})$ of assumptions and a conclusion $C$, we write $\mathcal{H} \vdash_{\mathcal{ND}_0} C$, iff there is an \linkterm{$\mathcal{ND}_0$ derivation}{c_derivation} \linkterm{tree}{tree} whose \linkterm{leaves}{tree} are in $\mathcal{H}$
}{nd0_derivation_tree}

\definition{Hypothetical Reasoning}{
Notice that the rule ($\Rightarrow I$) proves $A \Rightarrow B$ by exhibiting an \linkterm{$\mathcal{ND}_0$ derivation}{nd0_derivation_tree} $D$ (\textit{depicted by the three vertical dots}) of $B$ from a \textbf{local hypothesis} $A$. ($\Rightarrow I$) then \textbf{discharges} the local hypothesis (\textit{get rid of $A$, which can only be used in $D$}) and \linkterm{concludes}{inference_rules_logic} $A \Rightarrow B$. This mode of reasoning is called \textbf{hypothetical reasoning} (\linkterm{proof by local hypothesis}{proof_by_local_hypothesis}).
}{hypothetical_reasoning}

\commandnote{
The $\mathcal{TND}$ \linkterm{axiom}{axiom_logic} in the \linkterm{$\mathcal{ND}_0$ calculus}{nd_0} is called \textit{the law/principle of excluded middle/third} (in latin \textit{tertium non datur}) is only acceptable in \textbf{classical logic}, intuitionistic/constructive logic/mathematics does not affirm that law. (That applies for the derived rule ($\neg E$) too)
}


\begin{figure}[H]
    \centering
    \begin{tcolorbox}[colback=white, colframe=black, sharp corners, boxrule=0.5pt]
        \begin{multicols}{3}
            \noindent
            \begin{minipage}{\linewidth}   
                \[
                    \infer[\text{Ax}]{\Gamma, A \vdash A}{}
                \]
            \end{minipage}
            \noindent
            \begin{minipage}{\linewidth}   
                \[
                    \infer[\text{weaken}]{\Gamma, A \vdash B}{\Gamma \vdash B}
                \]
            \end{minipage}
            \noindent
            \begin{minipage}{\linewidth}   
                \[
                    \infer[\mathcal{TND}]{\Gamma \vdash A \lor \neg A}{}
                \]
            \end{minipage}
        \end{multicols}
        \vspace{1em}
        \begin{multicols}{3}
            \noindent
            \begin{minipage}{\linewidth}   
                \[
                    \infer[\land I]{\Gamma \vdash A \land B}{\Gamma \vdash A \qquad \Gamma \vdash B}
                \]
            \end{minipage}
            \noindent
            \begin{minipage}{\linewidth}   
                \[
                    \infer[\land E_l]{\Gamma \vdash A}{\Gamma \vdash A \land B}
                \]
            \end{minipage}
            \noindent
            \begin{minipage}{\linewidth}   
                \[
                    \infer[\land E_r]{\Gamma \vdash B}{\Gamma \vdash A \land B}
                \]
            \end{minipage}
        \end{multicols}
        \vspace{1em}
        \begin{multicols}{3}
            \noindent
            \begin{minipage}{\linewidth}   
                \[
                    \infer[\lor I_l]{\Gamma \vdash A \lor B}{\Gamma \vdash A}
                \]
            \end{minipage}
            \noindent
            \begin{minipage}{\linewidth}   
                \[
                    \infer[\lor E]{\Gamma \vdash C}{\Gamma \vdash A \lor B \qquad \Gamma, A \vdash C \qquad \Gamma, B \vdash C}
                \]
            \end{minipage}
            \noindent
            \begin{minipage}{\linewidth}   
                \[
                    \infer[\lor I_r]{\Gamma \vdash A \lor B}{\Gamma \vdash B}
                \]
            \end{minipage}
        \end{multicols}
        \vspace{1em}
        \begin{multicols}{2}
            \noindent
            \begin{minipage}{\linewidth}   
                \[
                    \infer[\Rightarrow I]{\Gamma \vdash A \Rightarrow B}{\Gamma, A \vdash B}
                \]
            \end{minipage}
            \noindent
            \begin{minipage}{\linewidth}   
                \[
                    \infer[\Rightarrow E]{\Gamma \vdash B}{\Gamma \vdash A \Rightarrow B \qquad \Gamma \vdash A}
                \]
            \end{minipage}
        \end{multicols}
        \vspace{1em}
        \begin{multicols}{2}
            \noindent
            \begin{minipage}{\linewidth}   
                \[
                    \infer[\neg I]{\Gamma \vdash \neg A}{\Gamma , A \vdash \bot}
                \]
            \end{minipage}
            \noindent
            \begin{minipage}{\linewidth}   
                \[
                    \infer[\neg E]{\Gamma \vdash A}{\Gamma \vdash \neg \neg A}
                \]
            \end{minipage}
        \end{multicols}
        \vspace{1em}
        \begin{multicols}{2}
            \noindent
            \begin{minipage}{\linewidth}   
                \[
                    \infer[\bot I]{\Gamma \vdash \bot}{\Gamma \vdash \neg A \qquad \Gamma \vdash A}
                \]
            \end{minipage}
        
            \noindent
            \begin{minipage}{\linewidth}   
                \[
                    \infer[\bot E]{\Gamma \vdash A}{\Gamma \vdash \bot}
                \]
            \end{minipage}
        \end{multicols}
    \end{tcolorbox}
    \caption{Propositional Sequent Style Natural Deduction Calculus}
    \label{fig:nd0_rules_seq}
\end{figure}

\commandnote{
The sequent style of \linkterm{$\mathcal{ND}_0$ calculus}{nd_0} is shown in \figref{fig:nd0_rules_seq} 
}

\textbf{Example: } \Figref{fig:gentzen_proof,fig:linear_proof,fig:sequent_proof,fig:fitch_proof} show different $\mathcal{ND}_0$ \linkterm{proof styles}{proofs} for the following \linkterm{propositional formula}{wff0_def}:
\[
    (A \lor B) \land (A \Rightarrow C) \land (B \Rightarrow C) \Rightarrow C
\]
\begin{figure}[H]
    \centering
    \begin{tcolorbox}[colback=white, colframe=black, sharp corners, boxrule=0.5pt]
        
        Let $\Gamma = (A \lor B) \land ((A \Rightarrow C) \land (B \Rightarrow C))$.
        \vspace{1em}
        
        $$
        \infer[\Rightarrow I]
        {\vdash \Gamma \Rightarrow C}
        {
            \infer[\lor E]
            {\Gamma \vdash C}
            {
                % Branch 1: Get A or B
                \infer[\land E_l]
                {\Gamma \vdash A \lor B}
                {
                    \infer[\text{Ax}]{\Gamma \vdash \Gamma}{}
                }
                &
                % Branch 2: Prove C assuming A
                \infer[\Rightarrow E]
                {\Gamma, A \vdash C}
                {
                    \infer[\land E_l]
                    {\Gamma, A \vdash A \Rightarrow C}
                    {
                        \infer[\land E_r]
                        {\Gamma, A \vdash (A \Rightarrow C) \land (B \Rightarrow C)}
                        {
                            \infer[\text{weaken}]
                            {\Gamma, A \vdash \Gamma}
                            {
                                \infer[\text{Ax}]{\Gamma \vdash \Gamma}{}
                            }
                        }
                    }
                    &
                    \infer[\text{Ax}]{\Gamma, A \vdash A}{}
                }
                &
                % Branch 3: Prove C assuming B
                \infer[\Rightarrow E]
                {\Gamma, B \vdash C}
                {
                    \infer[\land E_r]
                    {\Gamma, B \vdash B \Rightarrow C}
                    {
                        \infer[\land E_r]
                        {\Gamma, B \vdash (A \Rightarrow C) \land (B \Rightarrow C)}
                        {
                            \infer[\text{weaken}]
                            {\Gamma, B \vdash \Gamma}
                            {
                                \infer[\text{Ax}]{\Gamma \vdash \Gamma}{}
                            }
                        }
                    }
                    &
                    \infer[\text{Ax}]{\Gamma, B \vdash B}{}
                }
            }
        }
        $$
        
    \end{tcolorbox}
    \caption{Gentzen Sequent Style Proof}
    \label{fig:sequent_proof}
\end{figure}


\begin{figure}[H]
    \centering
    \begin{tcolorbox}[colback=white, colframe=black, sharp corners, boxrule=0.5pt]
        
        Let $\Gamma = (A \lor B) \land ((A \Rightarrow C) \land (B \Rightarrow C))$.
        \vspace{1em}
        
        $$
        \infer[\Rightarrow I^1]
        { \Gamma \Rightarrow C }
        {
            \infer[\lor E^2]
            { C }
            {
                % Branch 1: Isolate (A or B)
                \infer[\land E_l]
                { A \lor B }
                { [\Gamma]^1 }
                &
                % Branch 2: Case A
                \infer[\Rightarrow E]
                { C }
                {
                    \infer[\land E_l]
                    { A \Rightarrow C }
                    {
                        \infer[\land E_r]
                        { (A \Rightarrow C) \land (B \Rightarrow C) }
                        { [\Gamma]^1 }
                    }
                    &
                    [A]^2
                }
                &
                % Branch 3: Case B
                \infer[\Rightarrow E]
                { C }
                {
                    \infer[\land E_r]
                    { B \Rightarrow C }
                    {
                        \infer[\land E_r]
                        { (A \Rightarrow C) \land (B \Rightarrow C) }
                        { [\Gamma]^1 }
                    }
                    &
                    [B]^2
                }
            }
        }
        $$
        
    \end{tcolorbox}
    \caption{Gentzen-Prawitz Style Proof}
    \label{fig:gentzen_proof}
\end{figure}


\begin{figure}[H]
    \centering
    \begin{tcolorbox}[colback=white, colframe=black, sharp corners, boxrule=0.5pt]
        \renewcommand{\arraystretch}{1.3} 
        \begin{tabular}{r l l}
            \textbf{Step} & \textbf{Formula} & \textbf{Rule Applied} \\
            \hline
            (1) & $(A \lor B) \land (A \Rightarrow C) \land (B \Rightarrow C)$ & Assumption \\
            (2) & $A \lor B$ & $\land E_l$ (on 1) \\
            (3) & $(A \Rightarrow C) \land (B \Rightarrow C)$ & $\land E_r$ (on 1) \\
            (4) & $A \Rightarrow C$ & $\land E_l$ (on 3) \\
            (5) & $B \Rightarrow C$ & $\land E_r$ (on 3) \\
            \hline
            (6) & \quad $A$ & Assumption (Case 1) \\
            (7) & \quad $C$ & $\Rightarrow E$ (on 4 and 6) \\
            \hline
            (8) & \quad $B$ & Assumption (Case 2) \\
            (9) & \quad $C$ & $\Rightarrow E$ (on 5 and 8) \\
            \hline
            (10) & $\quad C$ & $\lor E$ (on 2, 7 and 9) \\
                 & & \textit{Discharges assumptions 6 \& 8} \\
            (11) & $((A \lor B) \land (A \Rightarrow C) \land (B \Rightarrow C)) \Rightarrow C$ & $\Rightarrow I$ (on 1 and 10) \\
                 & & \textit{Discharges assumption 1} \\
        \end{tabular}
    \end{tcolorbox}
    \caption{Linearized (Tabular) Style Proof}
    \label{fig:linear_proof}
\end{figure}

\begin{figure}[H]
    \centering
    \fbox{
        \begin{sdeduce}
            \hypl[1] (A \lor B) \land ((A \Rightarrow C) \land (B \Rightarrow C)) & Assumption \\
            \sub[1] A \lor B & $\land \text{E}_l$ (on 1) \\
            \sub[1] (A \Rightarrow C) \land (B \Rightarrow C) & $\land \text{E}_r$ (on 1) \\
            \sub[1] A \Rightarrow C & $\land \text{E}_l$ (on 3) \\
            \sub[1] B \Rightarrow C & $\land \text{E}_r$ (on 3) \\
            \hypl[2] A & Assumption \\
            \sub[2] C & $\Rightarrow \text{E}$ (on 4, 6) \\
            \hypl[2] B & Assumption \\
            \sub[2] C & $\Rightarrow \text{E}$ (on 5, 8) \\
            \sub[1] C & $\lor \text{E}$ (on 2, 7, 9) \\
            ((A \lor B) \land ((A \Rightarrow C) \land (B \Rightarrow C))) \Rightarrow C & $\Rightarrow \text{I}$ (on 1--10) \\
        \end{sdeduce}
    }
    \caption{Fitch Style Proof}
    \label{fig:fitch_proof}
\end{figure}

\definition{First-Order Signature}{
A First-Order (Logic) Signature (FOL Signature) is a tuple $\Sigma := \tuple{\Sigma^f, \Sigma^p}$ where:
\begin{itemize}
\item $\Sigma^f := \bigcup_{k \in \N} \Sigma_k^f$ of \textbf{function constants} (or \textbf{terms}), where members of $\Sigma_k^f$ denote the $k$-ary \linkterm{functions}{function} on \textbf{individuals}
\item $\Sigma^p := \bigcup_{k \in \N} \Sigma_k^p$ of \textbf{predicate constants}, where member of $\Sigma_k^p$ denote $k$-ary \linkterm{relations}{relation} among individuals
\item $\Sigma_k^f$ and $\Sigma_k^p$ are \linkterm{pairwise disjoint}{pairwise_disjoint}, \linkterm{countable}{countable} \linkterm{sets}{def:set} of symbols for each $k \in \N$.
\item A $0$-ary \textbf{function constant} refers to a single \textbf{individual}, therefore, we call it an \textbf{individual constant}
\end{itemize}
}{fol_sig}

\definition{Predicate Logic Without Quantifiers $\text{PL}^\text{nq}$}{
Given a \linkterm{FOL Signature}{fol_sig} $\Sigma$, the \textbf{formulae} of $\text{PL}^\text{nq}$ are given by the following \linkterm{grammar}{phrase_structure_grammar}:
\begin{align*}
f^k \quad &{\in} \quad \Sigma_k^f \\
p^k \quad &{\in} \quad \Sigma_p^f \\
t \quad &{:=} \quad f^0 \mid f^k(t_1, \cdots, t_k) \\
A \quad &{:=} \quad p^k(t_1, \cdots, t_k) \mid \neg A \mid A \land A
\end{align*}
}{plnq}

\definition{Well-Formed Terms}{
We denote the \linkterm{set}{def:set} of all well-formed \linkterm{terms}{fol_sig} over a \linkterm{FOL Signature}{fol_sig} $\Sigma$ with $\text{wff}_{\iota}(\Sigma)$, and the \linkterm{closed}{ground_pl0} ones with $\text{cwff}_{\iota}(\Sigma)$
}{wff_terms}

\definition{Well-Formed Formulae}{
We denote the \linkterm{set}{def:set} of all well-formed \linkterm{formulae}{fol_sig} over a \linkterm{FOL Signature}{fol_sig} $\Sigma$ with $\text{wff}_o(\Sigma)$, and the \linkterm{closed}{ground_pl0} ones with $\text{cwff}_o(\Sigma)$
}{wff_formulae_fol} 

\commandnote{
The Greek letter $\iota$ (read \textit{iota}) stands for \linkterm{individuals}{fol_sig} (objects), however, $o$ (read \textit{omicron}) refers to \linkterm{propositions}{proposition} that can be true of false
}

\definition{$\text{PL}^\text{nq}$ Universe}{
\linkterm{$\text{PL}^\text{nq}$}{plnq} uses the universe $\mathcal{D}_{\text{PL}^\text{nq}} := \mathcal{D}_0 \cup \mathcal{D}_{\iota}$ where $\mathcal{D}_0 = \set{T,F}$ is the \linkterm{set}{def:set} of truth values and $\mathcal{D}_{\iota} \neq \emptyset$ is a non-empty \linkterm{set}{def:set} of \linkterm{individuals}{fol_sig}
}{domain_plnq}

\definition{Interpretation in \linkterm{$\text{PL}^\text{nq}$}{plnq}}{
The interpretation function in \linkterm{$\text{PL}^\text{nq}$}{plnq} \linkterm{assigns}{var_ass_pl0} values to \linkterm{constants}{fol_sig}:
\begin{enumerate}
    \item We use the \linkterm{interpretation}{model_pl0} from \linkterm{$\text{PL}^0$}{def:pl0_as_ls}
        \begin{itemize}
        \item $\func{\mathcal{I}(\neg)}{\mathcal{D}_0}{\mathcal{D}_0}$; $T \mapsto F, F \mapsto T$
        \item $\func{\mathcal{I}(\land)}{\cartprod{\mathcal{D}_0,\mathcal{D}_0}}{\mathcal{D}_0}$; $\tuple{\alpha, \beta} \mapsto T$, iff $\alpha = \beta = T$
        \item $\mathcal{I}(\top) = T$
        \item $\mathcal{I}(\bot) = F$
        \end{itemize}
    \item We interpret \linkterm{individual constants}{fol_sig} as \linkterm{individuals}{fol_sig}:
    $
    \func{\mathcal{I}}{\Sigma_0^f}{\mathcal{D}_{\iota}}
    $
    \item We interpret \linkterm{function constants}{fol_sig} as \linkterm{functions}{function}:
    $
    \mathcal{I}: \Sigma_{k \neq 0}^f \to \mathcal{D}_{\iota}^{k \neq 0} \to \mathcal{D}_{\iota}
    $
    \item We interpret \linkterm{predicate constants}{fol_sig} as \linkterm{relations}{relation}:
    $
    \func{\mathcal{I}}{\Sigma_k^p}{\powerset{\mathcal{D}_{\iota}^{k}}}
    $
\end{enumerate}
}{interpretation_plnq}

\commandnote{
This definition naturally includes propositions as a special case. 
By defining $\func{\mathcal{I}}{\Sigma_k^p}{\powerset{\mathcal{D}_{\iota}^{k}}}$, we see that for $k=0$:
\[
\mathcal{D}_{\iota}^0 = \{\tuple{}\} \implies \powerset{\mathcal{D}_{\iota}^0} = \set{\emptyset, \set{\tuple{}}} \cong \set{F, T}
\]
Thus, a 0-ary predicate constant is interpreted as a truth value (a proposition), just like in propositional logic.
}

\definition{Value Function}{
The value function \linkterm{assigns}{var_ass_pl0} values to \linkterm{formulae}{wff0_def}:
\begin{enumerate}
\item $\mathcal{I}(f(A_1, \cdots, A_k)) := \mathcal{I}(f)(\mathcal{I}(A_1), \cdots, \mathcal{I}(A_k))$
\item $\mathcal{I}(p(A_1, \cdots, A_k)) := T, \text{ iff } \tuple{\mathcal{I}(A_1), \cdots, \mathcal{I}(A_k)} \in \mathcal{I}(p)$ 
\item and Just as in \linkterm{$\text{PL}^0$}{def:pl0_as_ls}:
\begin{itemize}
    \item $\mathcal{I}(\neg A) := \mathcal{I}(\neg) (\mathcal{I}(A))$
    \item $\mathcal{I} (A \land B) := \mathcal{I}(\land) (\mathcal{I}(A), \mathcal{I}(B))$
\end{itemize}
\end{enumerate}
}{value_function_plnq}

\Theorem{
    \linkterm{$\text{PL}^\text{nq}$}{plnq} is \textbf{isomorphic} to \linkterm{$\text{PL}^0$}{def:pl0_as_ls}
}{thm:plnq_pl0}

\Theorem{
\textbf{Unsatisfiability Theorem}. $\mathcal{H} \vDash A \Leftrightarrow \mathcal{H} \cup \set{\neg A}$ is \linkterm{unsatisfiable}{unsatisfiable_ls}.
}{thm:unsat}

\Proof{
\begin{enumerate}
\item Assume $H \vDash A$
\begin{itemize}
    \item For any $\varphi$ with $\varphi \vDash H$ we have $\varphi \vDash A$ and thus $\varphi \nvDash (\neg A)$
\end{itemize}
\item Assume $\mathcal{H} \cup \set{\neg A}$ is \linkterm{unsatisfiable}{unsatisfiable_ls}
\begin{itemize}
    \item For any $\varphi$ with $\varphi \vDash \mathcal{H}$ we have $\varphi \nvDash (\neg A)$ and thus $\varphi \vDash A$
\end{itemize} 
\end{enumerate}    
}





